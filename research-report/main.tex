\documentclass[11pt]{article}

% PRE-AMBLE --------------------------------------------------------------------------------------------------------------------------------

\usepackage[
    backend=biber,
    style=verbose % citation style
]{biblatex} % used for bibliographies/citations

\usepackage{dirtytalk} % used for quotations
\usepackage{graphicx}
\usepackage{makecell}
\usepackage{multirow}
\usepackage{parskip}

\title{Research for \say{Does the ability to simulate Newtonian Physics indicate we finally understand it?}}
\author{Blaire Villareal}
\date{May 2025}

\addbibresource{refs.bib}

% ACTUAL DOC ---------------------------------------------------------------------------------------------------------------------------------

\begin{document}

\maketitle

\raggedright % removes hyphenation

% ---------------------------------------------------------------

\section*{Introduction}

This paper serves as the research for my report on \say{Does the ability to simulate Newtonian Physics indicate we finally understand it?},
containing: 

\begin{enumerate}
    \item An overview on the workings of Newtonian Physics;
    \item Use-cases of Newtonian Physics in Simulations;
    \item The advantages/benefits of simulating Newtonian Physics;
    \item A discussion on whether we would be able to \say{\textit{master}} Newtonian Physics towards the future.
\end{enumerate}

\par 

In each section, it will contain both a reference to a relevant source and
a brief summary about why said reference would be useful in answering that sub-topic. Within these references will
include important pictures, figures and diagrams that will be crucial in illustrating specific key-points. 

\newpage

% ----------------------------------------------------------------

\section{Overview on Newtonian Physics}

% \lq Newtonian Physics\rq, also known as \lq Newtonian/Classical Mechanics\rq, is a foundational
% theory focused on describing the physical motion of objects. Traditionally, the study is divided into 3 
% separate subcategories:

% \begin{itemize}
%     \item \textbf{Statics}: the analysis of force (and torque) of objects with no acceleration;
%     \item \textbf{Dynamics}: the analysis of force (and torque) of objects including acceleration \parencite{kinematics};
%     \item \textbf{Kinematics}: the analysis of motion (and torque) of objects \parencite{kinematics}
% \end{itemize}

% \par

[1.] \parencite{classical_mechanics_intro} \par Course notes regarding the fundamental parts of 
what classical mechanics are, the associated sub-topics and related fundamental equations/diagrams.  

[2.] \parencite{kinematics} \par This paper explores in-depth on how kinematics and dynamics work in conjunction with one another,
the applications and the purpose of the \say{geometry of motion} within a wider scope.

[3.] \parencite{static_dynamic} \par This book goes deeper inside the \say{static} and \say{dynamic} fields of study,
going even further into the equations and notation. This will provide most of the information regarding the static-dynamic fields.

\section{Use-cases of Newtonian Physics in Simulations}

[1.] \parencite{neumann_compendium} \par A short compilation of both summaries and excerpts from Neumann's works,
it provides insight on why modelling using Newtonian Physics is beneficial, and where modelling using 
Newtonian Physics is invalid.

[2.] \parencite{beamng_softbody_physics} \par An IGN article which commentates over the revolutionary CryEngine softbody physics
back in 2014, this includes interviews on how BeamNG simulated realistic crashes and how softbody physics work.

[3.] \parencite{fifa_air_resistance} \par Another small article which briefly talks how fluid dynamics were simulated in the video-game franchise
\say{FIFA} in order to create more seamless and realistic ball movement.

[4.] \parencite{fluid_dynamic_workings} \par A scientific article talking about the history of fluid dynamics, as well as the separate applications
that came from it (primitive military applications to biofluid mechanics). 

[5.] \parencite{game_physics} \par A tutorial book detailing common physics and code used in games, and how are they made as such.
This provides a high-level outline on what code abstracts in order to simulate the idea of certain Newtonian concepts. 

\section{The advantages of simulating Newtonian Physics}

[1.] \parencite{physics_learning} \par An article/study on how computer physics simulations have helped
improve the teaching quality of physics lessons, and the positive correlation between physics simulations and declining failure rates.

[2.] \parencite{robot_kinematics} \par A book which demonstrates how Newtonian physics and modelling how it works can be put into practice,
with this example being through robot forward/reverse kinematics. This shows how Newtonian physics is still a foundational aspect of modern innovations 
and its importance to the future of an advanced society.

[3.] \parencite{newtonian_synthesis} \par An article which takes a more philosophical view on Newtonian Physics, and how it has changed the world fundamentally.
This will be useful to cite from both leading up to and during the next section involving \say{\textit{master}} over classical physics due to its critical view on the impact
of Newtonian Mechanics.

\newpage

% ----------------------------------------------------------------

% table example here
% \begin{table}[h!]
%     \centering
%     \begin{tabular}{ |c|c|c|c|c|c|c|c|c| }
%         \hline
%             \multirow{2}{4em}{ \quad } & \multicolumn{4}{ |c| }{CLASS A: Percentage of correct answers} & \multicolumn{4}{ |c| }{CLASS B: Percentage of correct answers} \\
%         \cline{2-9}
%             & X & Y & Z & W & Z & Z & Z & Z \\
%         \hline
%             X & Y & Z & W & Z & Z & Z & Z & U \\
%         \hline
%     \end{tabular}
% \end{table}

\printbibliography

\end{document}
