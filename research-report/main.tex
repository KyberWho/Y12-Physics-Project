\documentclass[12pt]{article}

% PRE-AMBLE --------------------------------------------------------------------------------------------------------------------------------

\usepackage[
    backend=biber,
    style=verbose % citation style
]{biblatex} % used for bibliographies/citations

\usepackage{dirtytalk} % used for quotations
\usepackage{graphicx}
\usepackage{parskip}

\title{Research for \say{Does simulating Newtonian Physics mean we finally understand it?}}
\author{Blaire Villareal}
\date{April 2025}

\addbibresource{refs.bib}

% ACTUAL DOC ---------------------------------------------------------------------------------------------------------------------------------

\begin{document}

\maketitle

\raggedright % removes hyphenation

% ---------------------------------------------------------------

\section*{Introduction}

This paper serves as the research for my report on "Does simulating Newtonian Physics mean we finally understand it?",
containing: 

\begin{enumerate}
    \item An overview on the workings of Newtonian Physics;
    \item Use-cases of Newtonian Physics;
    \item The advantages/benefits of simulating Newtonian Physics;
    \item A discussion on whether we could \say{\textit{master}} Newtonian Physics towards the future.
\end{enumerate}

\par 

In each section, it will contain both a reference to a relevant source and
a brief summary about why said reference would be useful in answering that sub-topic.

\newpage

% ----------------------------------------------------------------

\section{Overview on Newtonian Physics}

% \lq Newtonian Physics\rq, also known as \lq Newtonian/Classical Mechanics\rq, is a foundational
% theory focused on describing the physical motion of objects. Traditionally, the study is divided into 3 
% separate subcategories:

% \begin{itemize}
%     \item \textbf{Statics}: the analysis of force (and torque) of objects with no acceleration;
%     \item \textbf{Dynamics}: the analysis of force (and torque) of objects including acceleration \parencite{kinematics};
%     \item \textbf{Kinematics}: the analysis of motion (and torque) of objects \parencite{kinematics}
% \end{itemize}

% \par

[1.] \parencite{classical_mechanics_intro} \par Course notes regarding the fundamental parts of 
what classical mechanics are, the associated sub-topics and related fundamental equations/diagrams.  

[2.] \parencite{kinematics} \par This paper explores in-depth on how kinematics and dynamics work in conjunction with one another,
the applications and the purpose of the \say{geometry of motion} within a wider scope.

[3.] \parencite{static_dynamic} \par This book goes deeper inside the \say{static} and \say{dynamic} fields of study,
going even further into the equations and notation. This will provide most of the information regarding the static-dynamic fields.

\section{Use-cases of Newtonian Physics}

[1.] \parencite{neumann_compendium} \par A short compilation of both summaries and excerpts from Neumann's works,
it provides insight on why modelling using Newtonian Physics is beneficial, and where modelling using 
Newtonian Physics is invalid.

[2.] \parencite{beamng_softbody_physics} \par An IGN article which commentates over the revolutionary CryEngine softbody physics
back in 2014, this includes interviews on how BeamNG simulated realistic crashes and how softbody physics work.

[3.] \parencite{particle_system_manual} \par A detailed guide created early in the early 2000s, this article explores how a particle
system is made, which is especially useful for simulating fluid dynamics (e.g water waves, smoke physics, etc.).

[4.] \parencite{ragdoll_physics} \par An article which talks about the problems with recreating ragdoll physics
from the programming perspective. This is useful to showcase an example on what goes on behind the scenes 
when recreating Newtonian Physics.

[4i.] \parencite{naturalmotion_euphoria} \par A small article which focuses on the \say{Euphoria} engine: 
dynamic animations which interact accordingly to the environment and subsequent forces acting on it. This is useful to showcase
another similar yet unique way Newtonian Physics are used in simulations.

[5.] \parencite{fifa_air_resistance} \par Another small article which briefly talks how fluid dynamics were simulated in the video-game franchise
\say{FIFA} in order to create more seamless and realistic ball movement.

[6.] \parencite{fluid_dynamic_workings} \par A scientific article talking about the history of fluid dynamics, as well as the separate applications
that came from it (primitive military applications to biofluid mechanics). 

% needs better grammar
[7.] \parencite{game_physics} \par A tutorial book detailing common physics and code used in games, and how are they made as such.
This provides a high-level overview on what physics models abstract in order to showcase something specifically. 

\section{The advantages of simulating Newtonian Physics}

[1.] \parencite{physics_learning} \par An article/study on how computer physics simulations have helped
improve the teaching quality of physics lessons, and the positive correlation between physics simulations and declining failure rates. 

[2.] \parencite{robot_kinematics} \par A book which demonstrates how Newtonian physics and modelling how it works can be put into practice,
with this example being through robot forward/reverse kinematics.

[3.] \parencite{newtonian_synthesis} \par An article which takes a more philosophical view on Newtonian Physics, and how it has changed the world fundamentally.
This will be useful to cite from both leading up to and during the next section involving \say{\textit{master}} over classical physics due to its critical view on the impact
of Newtonian Mechanics.

\newpage

% ----------------------------------------------------------------

\printbibliography

\end{document}
