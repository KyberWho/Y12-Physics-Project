\documentclass[12pt]{article}

% PRE-AMBLE --------------------------------------------------------------------------------------------------------------------------------

\usepackage[
    backend=biber,
    style=verbose % citation style
]{biblatex} % used for bibliographies/citations

\usepackage{dirtytalk} % used for quotations
\usepackage{graphicx}
\usepackage{parskip}

\title{Research for \say{Does simulating Newtonian Physics mean we finally understand it?}}
\author{Blaire Villareal}
\date{April 2025}

\addbibresource{refs.bib}

% ACTUAL DOC ---------------------------------------------------------------------------------------------------------------------------------

\begin{document}

\maketitle

\raggedright % removes hyphenation

% ---------------------------------------------------------------

\section*{Introduction}

This paper serves as the research for my report on "Does simulating Newtonian Physics mean we finally understand it?",
containing: 

\begin{enumerate}
    \item An overview on the workings of Newtonian Physics \parencite{einstein};
    \item Use-cases of Newtonian Physics;
    \item The advantages/benefits of simulating Newtonian Physics;
    \item A discussion on whether we could \say{\textit{master}} Newtonian Physics towards the future.
\end{enumerate}

\par 

In each section, it will contain both a reference and a brief summary about why said reference
would be useful in answering that sub-topic.

\newpage

% ----------------------------------------------------------------

\section{Overview on Newtonian Physics}

% \lq Newtonian Physics\rq, also known as \lq Newtonian/Classical Mechanics\rq, is a foundational
% theory focused on describing the physical motion of objects. Traditionally, the study is divided into 3 
% separate subcategories:

% \begin{itemize}
%     \item \textbf{Statics}: the analysis of force (and torque) of objects with no acceleration;
%     \item \textbf{Dynamics}: the analysis of force (and torque) of objects including acceleration \parencite{kinematics};
%     \item \textbf{Kinematics}: the analysis of motion (and torque) of objects \parencite{kinematics}
% \end{itemize}

% \par

[1.] \parencite{kinematics} \par This paper explores in-depth on how kinematics and dynamics work in conjunction with one another,
the applications and the purpose of the \say{geometry of motion} within a wider scope.

[2.]aa

% ----------------------------------------------------------------

\printbibliography

\end{document}
